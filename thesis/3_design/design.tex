% this file is called up by thesis.tex
% content in this file will be fed into the main document

\chapter{Design} % top level followed by section, subsection


% ----------------------- paths to graphics ------------------------

% change according to folder and file names
\ifpdf
    \graphicspath{{7/figures/PNG/}{7/figures/PDF/}{7/figures/}}
\else
    \graphicspath{{7/figures/EPS/}{7/figures/}}
\fi


% ----------------------- contents from here ------------------------
% 

\section{Requirements}

As mentioned our research questions are divided to address both design and
evaluation separately. In addition, the design requirements can be further
divided into those addressing the overall filesystem, the surrounding
framework and computational offloading. The resulting categorization is shown
below.

\begin{itemize}
    \item Framework
    \begin{enumerate}
        \item What techniques can be used to minimize the complexity required to
            use a hybrid LFS?
        \item What mechanisms allow for simplified replacement of used existing
            technologies in a hybrid LFS?
    \end{enumerate}
    \item Filesystem
    \begin{enumerate}
        \item What existing technologies are best suited for a hybrid LFS?
    \end{enumerate}
    \item Offloading
    \begin{enumerate}
        \item How to register CSx compute kernels using existing operating
            system APIs?
        \item How to differentiate individual users, files and I/O operations in
            relation to their CSx compute kernel?
        \item How to design a compute kernel API that can be reused across devices?
        \item How to ensure user submitted CSx compute kernels are safe?
    \end{enumerate}
\end{itemize}

\subsection{Framework}

Traditional CSx prototypes often have a high barrier to entree due to
complicated dependencies that can not easily be installed through package
managers or software repositories. Our work directly addresses that minimizing
the complexity as design requirement as well as designing for simple
replacement of used technologies.

Firstly, all software dependencies must be installed isolated from system
dependencies. This is to prevent the host system being influenced by the use
of the prototype. Secondly, as many as possible software dependencies must be
userspace programs such that even users without administrative system
privileges can use it. Finally, the prototype must be able to run in a virtual
environment that allows for the emulation of ZNS SSDs. This is to overcome
hardware requirements given the lack of general availability for ZNS SSDs.

Our design achieves simple replacement of technologies by creating abstract
interfaces for them in a generalized form. Each of the custom software
components developed is only coupled to these interfaces. Should the need arise
than the implementation of these interfaces can be switched even without
recompilation of the dependent software component. In this work we will call
each of these software components \textit{modules}.

\subsection{Filesystem}

Until now the integration of filesystem support in CSxs has been a difficult
avenue. Regardsless, works such as Metal FS \cite{10.1145/3415580} and INSIDER
\cite{234968} definitely achieve some form of file system integration. However,
filesystem integration can still be substantially improved as will be shown in
this work. To do so we design our filesystem by selecting the best existing
technologies that achieve this.

Firstly, the design must be a \textit{Log-Structured Filesystem} (LFS) as that
easily enables journaling, snapshotting and recovery. While the implementation
of journaling and recovery are optional, snapshotting is essential for our
design. Secondly, the filesystem must be \textit{host-managed}, meaning that
wear-leveling and garbage-collection are managed by the filesystem on the host.
This will eliminate any logical to physical block address translation on the
device which would result in additional complexity and runtime overhead for
offloaded compute kernels. These differences between the host and device view
are often referred to as the \textit{semantic gap}. Third, the entirety of a
file, the inode, data locations and details of the I/O request must be easily
representable in memory. Such a memory representation of a file can then be
exposed to computational storage kernels. Lastly, access to snapshots must be
independent from regular file access to the extend that they can happen
concurrently otherwise evaluating any performance gains from such a prototype
becomes meaningless.

\subsection{Offloading}

Many recent CSx works still require changes to existing operating system
APIs\footnotemark[9] to operate. INSIDER \cite{234968} partly addressed this by
copying the signatures of common \textit{Portable Operating System Interface}
(POSIX) I/O calls and creating additional clones for offloading. While this
reduces complexity due to common familiarity and thorough understanding of
this pre-existing API, it still adds additional calls that must specifically be
targeted to support offloading. Metal FS \cite{10.1145/3415580} does not need
additional APIs to function instead changing UNIX pipe behavior. However, it is
not specified if this eliminates regular use of UNIX pipes on their filesystem
implementation. Our design manages to only use existing APIs found in almost all
\textit{UNIX derivates} further known as \textit{NIX}. The exact process of
offloading is achieved through the following requirements.

\footnotetext[9]{In the context of operating systems the term API and ABI is
used interchangeably although most implemenations solely use an ABI. A common
exception is microkernels that use usermode services to provide the majority of
features.}

Firstly, the compute kernels are registered through filesystem extended
attributes, commonly managed with the \textit{xattr} command on NIX operating
systems. Secondly, the calling \textit{Process IDentifier} (PID) must be
maintained in memory alongside the file and kernel upon such an extended
attributes call. In addition, storing this state in a non-persistent manner is
essential to the design. Lastly, the offloading behavior must be isolated to the
process setting the extended attribute. In other terms, regular accesses must
be unaffected by processes setting these extended attributes.

The device design requires that one universal ISA is used for offloading
execution. This eliminates users having to rewrite or recompile their compute
kernels across different vendors. In addition, one API needs to be defined in
such a way that it can be implemented by the vendor. The API needs to be
simple enough for users to use as well as for vendors to implement. However, the
vendor API code can not be linked at compile time as this would result in users
still having to recompile code across vendors.

Instead, an ISA is needed that can easily be executed by a virtual machine. In
addition, the ISA should support \textit{call} instructions and it is these
instructions that should \textit{trap} into the vendor API. Such an interface
that is defined at the binary level is known as an ABI.

Finally, the ISA should allow for safety features including but not limited to 
memory access validation, static verification and termination determiniation.
This will provide protection against misbehaving and malicious user code.

Optionally, compiled kernels should allow for variable relocation so that the
filesystem can reconfigure global variables on user submitted kernels. This
will greatly simplify static verification. However. the ability to support this
depends largely on the ISA.

Clearly, the overall design result is a cohesive solution with clear design
requirements. However, it took several iterations of the design process to
reach these requirements. Each of these iterations is described after the
implemenation section.

% ---------------------------------------------------------------------------
% ----------------------- end of thesis sub-document ------------------------
% ---------------------------------------------------------------------------