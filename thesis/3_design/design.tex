% this file is called up by thesis.tex
% content in this file will be fed into the main document

\chapter{Design} % top level followed by section, subsection


% ----------------------- paths to graphics ------------------------

% change according to folder and file names
\ifpdf
    \graphicspath{{7/figures/PNG/}{7/figures/PDF/}{7/figures/}}
\else
    \graphicspath{{7/figures/EPS/}{7/figures/}}
\fi


% ----------------------- contents from here ------------------------
% 

\section{Requirements}

\subsection{Framework}

% mmap shared memory vs monolith (iteration 1)
% accelerator API vs NVMe namespace (iteration 4)

\subsection{Filesystem}

% rtl_next vs FUSE (iteration 3)

% Two write pointers, one for RANDOM ZONE and one for LOG ZONE. Use of ZNS is
% optional but allows for lower write-amplification and more explicit garbage
% collection

\subsubsection{Concurrency}

\subsection{Offloading}

% Filesystem extended attributes, PID + INODE

\section{Iterations}

Before covering our overall design decisions we describe the iterations that
have occured during the design process. We cover this separately as to not
clutter the overall design and implementation sections. These next sections
describes the design requirements derived from the research questions, followed
by, the final design. The implementation details of the design are covered in
a subsequent chapter.

The overall design process consisted of four destinct iterations. Two relating
primarily to the framework, onr relating to the filesystem and one relating to
offloading. This section describes each iteration briefly before going in more
detail. Firstly regarding framework iterations we see the distinct decision
to switch from a multi process architecture using mmap for shared memory maps to
a monolithic application. The second iteration led to switching away from the
design of an accelerator API, much like Vulkan or OpenCL, to an artificial
extension of the NVMe namespace. The third iteration changed the use
rtld\_next \cite{rtldnext} to using a practical filesystem with FUSE. Lastly, as
computational storage API our work switched from using
\textit{Portable Operating System Interface} (POSIX) fadvise \cite{fadvise} to
extended attributes.

For each of these four iterations the advantages of the change as well as the
major issues with the previous solution are described. Each iteration is
described in the same order as previously defined.

\subsection{Shared Memory Monolith}

Modern operating systems offer fastly different methods to write sofware. From
kernel modules, to distributed processes, UNIX pipes and shared memory maps.
Choosing the right model impacts practicalities such as the amount of
development effort required and the robustness of the final solution. 
Furtermore, depending on the software architecture some solutions will be better
suited than others.

During the design the use of shared memory maps was replaced with using regular
shared memory. While both solutions provide shared memory they are fundamentally
different. A shared memory map is a file, leveraging the well known UNIX
principle \textit{everything is a file}, that allows two or more processes to
share a region of memory. While regular shared memory is limited to a single
process although it could share this memory with additional threads.

this single process shared memory solution is one of the most common found in
software today. The concepts of such a program are very well understood as
well as the development using imperative languages being straightforward.

While shared memory maps are typically found in device drivers, such as those
for graphics cards, their use consistutes severe additional development effort.
In conjuction with our design being a simulation there is no scientific value in
using shared memory maps for our design.

\subsection{NVMe Namespace Command Set}

\subsection{FUSE Filesystem}

\subsection{Extended Attributes}

% ---------------------------------------------------------------------------
% ----------------------- end of thesis sub-document ------------------------
% ---------------------------------------------------------------------------