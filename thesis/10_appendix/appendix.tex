% this file is called up by thesis.tex
% content in this file will be fed into the main document

%: ----------------------- name of chapter  -------------------------
\chapter*{Appendix} % top level followed by section, subsection
\addcontentsline{toc}{chapter}{\protect\numberline{}Appendix}%

% \appendix
\section{Artifact Description: Prototype emulation framework with
filesystem supporting hybrid access}

%%%%%%%%%%%%%%%%%%%%%%%%%%%%%%%%%%%%%%%%%%%%%%%%%%%%%%%%%%%%%%%%%%%%%
\subsection{Abstract}

This abstract contains the necessary information to install our functional
prototype, use it potentially rerunning our experiments and generate the
graphs as shown in our work.

\subsection{Artifact check-list (meta-information)}

{\small
\begin{itemize}
  \item {\textbf{Algorithm:} Shannon entropy}
  \item {\textbf{Program:} Python, BASH and fio, preinstalled in supplied
  QEMU image.}
  \item {\textbf{Compilation:} See table \ref{table:supportdependencies} for
  compilation dependencies, note that QEMU can be automatically compiled by our
  framework.}
  \item {\textbf{Data set:} Randomly generated input data using \textit{\textbackslash dev\textbackslash random}}
  \item {\textbf{Run-time environment:} virtualenv \& QEMU using Arch based Linux OS.}
  \item {\textbf{Hardware:} Intel Xeon Silver 4210 with 32GiB of working memory and
  a western digital WZS4C8T4TDSP303 ZNS SSD.}
  \item {\textbf{Metrics:} Execution time, throughput (MiB/S), data movement and
  host CPU load (time spent in wait vs sys / user)}
  \item {\textbf{Output:} Throughput and iops stored in csv files found in the
  measurements directory of our source code \cite{qemu-csd}}
  \item {\textbf{Experiments:} Scripts are provided to run each experiment some
  manual setup is required to configure the ZNS device.}
  \item {\textbf{How much disk space required (approximately)?:} 75 GB}
  \item {\textbf{How much time is needed to prepare workflow (approximately)?:} 4 hours}
  \item {\textbf{How much time is needed to complete experiments (approximately)?:} 2 days}
  \item {\textbf{Publicly available?:} Yes all code including unit tests, document
  source code (LaTeX), raw measurements and scripts to generate plots and graphs
  are publicly available at Github \cite{qemu-csd}}
  \item {\textbf{Code licenses?:} Apache 2}
  \item {\textbf{Data licenses?:} Public Domain}
\end{itemize}
}

\subsection{Description}

\subsubsection*{How to access}

Publicly available online \cite{qemu-csd}.

\subsubsection*{Hardware dependencies}

Our compilation process uses compiler arguments to optimize for the target CPU
architecture (\textit{march=native}) a different CPU microarchitecture could
result in fastly different results. In addition, the use of jitting with uBPF
requires that the host is using the x86 architecture.

\subsubsection*{Software dependencies}

Several compilers as well as build systems are required to compile QEMU. Once
QEMU is compiled the supplied image will have all dependencies preinstalled.

\begin{itemize}
    \item To compile QEMU at least GCC, CMake, meson, pyelftools and ninja need
    to be installed. Special CMake arguments are needed to disable default build
    targets that have additional dependencies. See our README for details
    \cite{qemu-csd}
    \item Native execution of our framework and filesytem requires the
    dependencies as specified in table \ref{table:supportdependencies}.
    Through CMake arguments several dependencies such as those for testing and
    documentation can be removed, see our README for details \cite{qemu-csd}.
    \item All other dependencies such as those listed in table
    \ref{table:operationdepdencies} are automatically installed by our
    framework. A modified virtualenv is used to configure the environment
    variables to expose these dependencies.
\end{itemize}

\subsubsection*{Data sets}

All data sets used to perform experiments where randomly generated using
\textit{\textbackslash dev\textbackslash random}. No data sets are stored
persistently. All scripts to perform experiments automatically generate the
required tests files and populate them with random data.

%%%%%%%%%%%%%%%%%%%%%%%%%%%%%%%%%%%%%%%%%%%%%%%%%%%%%%%%%%%%%%%%%%%%%
\subsection{Installation}

Installation can be executed using the following steps. There are many options
available such as native installation or using QEMU. These steps here only
cover installation using QEMU see our README for details on other methods
\cite{qemu-csd}. These installation steps assume all required software
dependencies are installed.

\bashexternal{resources/bash/install.sh}

%%%%%%%%%%%%%%%%%%%%%%%%%%%%%%%%%%%%%%%%%%%%%%%%%%%%%%%%%%%%%%%%%%%%%
\subsection{Experiment workflow}

Experiments are divided into two groups one for F2FS evaluations the other for
our emulation framework. Before continuing we describe several generic steps
required to passthrough the appropriate ZNS device to QEMU:

\bashexternal{resources/bash/vfio.sh}

Afterwards modify \textit{qemu-start-real-zns.sh} in the build directory
(available after installation). Ensure that the \textit{host=} parameter for
vfio-pci device matches the pci address of your ZNS device.

With the virtualenv activated, (\textit{source activate}), the QEMU VM can now
be started with ZNS device passthrough \textit{ld-sudo ./qemu-start-zns-real.sh}

In the absense of a real ZNS device use \textit{ld-sudo ./qemu-start-256-kvm.sh}
instead. However, the further workflow description is based on using an actual
ZNS device.

All subsequent commands shown are assumed to be executed after using SSH to
login to the QEMU VM. In addition all commands shown are executed relative
to root of the project directory unless otherwise specified.

Firstly the ZNS device needs to be configured with 4096 byte sectors. In
addition, we must assign the first namespace as conventional block storage.
Without such a conventional block storage parition an F2FS filesystem can not
be created.

Initially, delete all currently created namespaces:

\bashexternal{resources/bash/delete-namespace.sh}

Next create \& attach a 4GiB conventional block storage namespace and a
~800GiB ZNS namespace:

\bashexternal{resources/bash/create-namespace-f2fs.sh}

With the ZNS namespaces configured, an F2FS partition can be created and
mounted:

\bashexternal{resources/bash/format-f2fs.sh}

The F2FS filesystem must be mounted under the \textit{/mnt} directory prior to
starting any of the experiments. Afterwards, with the virtualenv activated, any
script can be started. These scripts are found in the
\textit{/scripts/benchmark} folder of the project \cite{qemu-csd} and include:

\begin{itemize}
    \item ld-sudo ./scripts/benchmarks/fio-read-write/fio-rand-performance-f2fs.sh
    \item ld-sudo ./scripts/benchmarks/fio-read-write/fio-seq-performance-f2fs.sh
\end{itemize}

Each of these benchmarks will automatically generate csv files with appropriate
names. These names match the convention found in our measurements folder.

The device paths used in these examples and the scripts inside the benchmark
folder need to be changed from \textit{\textbackslash dev\textbackslash nvme0}
in case QEMU is not used. In additon, the namespace configuration as
conventional block device is only supported when using the WZS4C8T4TDSP303 ZNS
drive.

FluffleFS workflow is similar but requires only a single namespace:

\bashexternal{resources/bash/delete-namespace.sh}

\bashexternal{resources/bash/create-namespace-fluffle.sh}

Since Fluffle uses userspace SPDK NVMe drivers the current kernel drivers needs
to be unloaded, after this commands such as \textit{nvme list} no longer work:

\bashexternal{resources/bash/setup-spdk.sh}

With SPDK configured the necessary paths and directories can now be setup:

\bashexternal{resources/bash/setup-fluffle.sh}

Finally, the experiments for FluffleFS can be run, the filesystem will
automatically be formatted between each measurement. The list of available
experiments includes. Note that these expirements must be started from the
\textit{/build/qemu-csd} directory:

\begin{itemize}
    \item ld-sudo ../../scripts/benchmarks/fio-read-write/fio-rand-performance-fluffle.sh
    \item ld-sudo ../../scripts/benchmarks/fio-read-write/fio-seq-performance-fluffle.sh
    \item ld-sudo ../../scripts/benchmarks/fio-read-write/fio-seq-concurrent-fluffle.sh
    \item ld-sudo ../../scripts/benchmarks/entropy-kernel.sh
    \item ld-sudo ../../scripts/benchmarks/entropy-regular.sh
    \item ld-sudo ../../scripts/benchmarks/passthrough-concurrent.sh
    \item ld-sudo ../../scripts/benchmarks/passthrough-regular.sh
\end{itemize}

Several of these experiments especially those for random read/write performance
can take a very long time to complete. Between experiments it is important to
validate that SPDK is functioning correctly as we routinely observed errors
during our evaluation.

This can be verified either waiting until an experiment completes or by killing
the BASH process and unmounting the filesystem. Next use the following command
to verify SPDK is still working:

\bashexternal{resources/bash/check-spdk.sh}

SPDK should start correctly and appear mounted under the specified test
directory. If SPDK fails to start the QEMU VM needs to be rebooted and the SPDK
setup needs to be repeated after reboot. In addition, do not forget to
reactivate the virtualenv inside the \textit{build/qemu-csd} directory after
every reboot.

%%%%%%%%%%%%%%%%%%%%%%%%%%%%%%%%%%%%%%%%%%%%%%%%%%%%%%%%%%%%%%%%%%%%%
\subsection{Evaluation and expected results}

Our performance evalation compares throughput in MiB/S and iops in several
microbenchmarks using sequential read and write and random read and write jobs.
These microbenchmarks are orchestrated by BASH script that take care of starting
and stopping the filesystem as well as calling \textit{fio}. Fio is a readily
available and industry standard tool to evaluate filesystem performance.

Apart from performance we measure data movement and host CPU load using a
shannon entropy algorithm. This algorithm is implemented twice, once in Python
using regular I/O and once in eBPF using offloaded I/O. The amount of data
movement is compared statically through mathematical computation while the host
CPU load is evaluated using \textit{time}. This readily available program is
able to categorize cpu time of a process across wait, userspace and kernelspace
states. The files for which the shannon entropy are filled with randomly
generated data using \textit{\textbackslash dev\textbackslash random}.

%%%%%%%%%%%%%%%%%%%%%%%%%%%%%%%%%%%%%%%%%%%%%%%%%%%%%%%%%%%%%%%%%%%%%
\subsection{Notes}

Please see our Github for more detailed instructions including a tutorial
video \cite{qemu-csd}.

% ---------------------------------------------------------------------------
%: ----------------------- end of thesis sub-document ------------------------
% ---------------------------------------------------------------------------

