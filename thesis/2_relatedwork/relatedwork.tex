% this file is called up by thesis.tex
% content in this file will be fed into the main document

\chapter{Related Work} % top level followed by section, subsection


% ----------------------- paths to graphics ------------------------

% change according to folder and file names
\ifpdf
    \graphicspath{{7/figures/PNG/}{7/figures/PDF/}{7/figures/}}
\else
    \graphicspath{{7/figures/EPS/}{7/figures/}}
\fi


% ----------------------- contents from here ------------------------
% 

While OpenCSD is unique it is by no means the first CSx prototype as there
has been over a decade of research in this field \cite{lukken2021past}. In this
decade we have seen a large variety of approaches, problem spaces, hardware
platforms and software APIs. While this decade of research shows overall
progression several important challenges are still open research
questions \cite{barbalacecomputational}.

In this section we first introduce characterstics of CSx prototypes such as the
hardware platform and software API. Secondly, introduce the most prominent open
research questions in CSx research. Third, provide an overview of notable works
of the past decade. Fourth, describe the general progression of the field across
these works. Lastly, describe fundamentally missing features that hinder
adoption and widespread use of CSxs.

% Introduce early working prototypes

% Demonstrate different hardware platforms, embedded CPU, vs FPGA.

% List of BPF using storage literature (Blockndp:, Extos: Data-centric
%extensible os, Ex-tension framework for file systems in user space, 
%  Safe and efficient remote application code execu-tion on disaggregated NVM,
% BPF for storage: an exokernel-inspired approach )

% Introduce previous ZCSD table but extended

\section{CSx characterstics}

There are four types of characteristics that set CSxs apart. First there is
the end user \textit{programming model}, the model of programming developers
have to use in order to interact with CSx. Second, there is the execution
environment, the level of software abstraction the end user programs are run on
(on the device). Third, is the \textit{degree of programmability}, the amount of
programming control offered to the end user. Lastly, there is the interface used
to connect the device to the host.

With end user programming models there are, in no particular order,
\textit{Dataflow (MapReduce, DAG)}, \textit{Client / Server (RPC, HTTP)},
\textit{Shared memory} and \textit{Declarative (Regex, MySQL)}. Although simple
categories, the correct attribution of these categories can become quite
complicated due to nuances. For instance, if an end user needs to link to a new
shared library to call methods with MySQL queries as argument, what should the
programming model be? In this work we argue \textit{Shared memory} as the
library that needs to be linked is introduced in the work.

In \textit{Computational Storage Execution Environments} (CSEE) we effectively
observe seven distinct levels of abstraction. These range from
\textit{Register-Transfer Level} (RTL) design such as using languages like
Verilog or VHDL to virtual machines. The compiled output of RTL designs is known
as a bitstream and typically programmed into an FPGA at runtime. The distinct
levels are listed below.

\begin{enumerate}
    \item Bistream; bitstream programmed directly unto FPGA.
    \item Embedded; single program, single memory space, no OS just 'real mode'.
    \item Accelerators; OpenCL, Vulkan.
    \item Real-Time operating system;
    \item Operating System;
    \item Container;
    \item Virtual Machine;
\end{enumerate}

We see a similarly large range in different \textit{degrees of programmability}
from no end user programmability at all such as with transparent operations to
arbitrary code execution. The six distinct levels are shown below. It
should be noted howerver, that the first two levels can be regarded as no or a
lack of end user programmability.

\begin{enumerate}
    \item Transparent operations, (de)compression (Playstation 5 I/O Controller)
    \item Fixed functions, unchangeable, workload specific \cite{2013-fast-active-flash}
    \item Fixed function dataflow programming \cite{Wickremesinghe02distributedcomputing}
    \item Query offloading, SQL, NoSql, Regex \cite{10.14778/2994509.2994512}
    \item Event driven (hooks) user-programmable functions \cite{10.1145/3429357.3430519}
    \item Arbitrary code execution, VHDL, eBPF, TCL \cite{10.1145/605432.605425, kourtis2020safe}
\end{enumerate}

It is important to illustrate that the \textit{degree of programmability} is
fundamentally different from the end user \textit{programming model} as one
denotes how programs are submitted and run on the device while the other denotes
what model an end user must use to develop for the CSx respectively.

\section{Past Works}

\begin{table}[H]
    \caption{Related CSx works from the past decade}
    \centering
    \begin{adjustbox}{width=1\textwidth}
        \begin{threeparttable}[]
            \begin{tabular}{lllll}
                \toprule
                \textbf{Name} & \textbf{Programmability} & \textbf{Programming Model} & \textbf{Interface} & \textbf{CSEE} \\
                \midrule
                Active Flash \cite{active-flash-piller, 2013-fast-active-flash} & Fixed functions & N.A. & SATA (OpenSSD) & Embedded \\
                Caribou \cite{10.14778/3137628.3137632} & Fixed functions (key-value store) & Client / Server (RPC) & Ethernet & Bitstream \\
                LeapIO \cite{10.1145/3373376.3378531} & Fixed functions & Transparent & Ethernet (RDMA) & Embedded \\
                CSD 2000 \cite{10.1145/3399666.3399934} & Fixed functions (compression) & Transparent & PCIe (NVMe) & Bitstream \\
                Active SSD \cite{6062973} & Event driven & Dataflow (streams) & PCIe & Operating system (Custom) \\
                Smart SSD \cite{6558444} & Event driven & Dataflow (MapReduce) & SATA & Embedded \\
                Smart SSD \cite{10.1145/2463676.2465295} & Event driven & Shared memory & SATA & Embedded \\
                Intelligent SSD \cite{10.1145/2464996.2465003, 10.1145/2505515.2507847} & Arbitrary code execution\footnotemark[7] & Shared memory\footnotemark[7] & N.A. & Operating system (Linux)\footnotemark[7] \\
                Ibex \cite{10.14778/2732967.2732972} & Query offloading (MySQL) & Declarative & SATA & Bitstream \\
                Willow \cite{186149} & Arbitrary code execution & Client / Server (RPC) & PCIe (NVMe) & Operating system (Custom) \\
                Biscuit \cite{2016-isca-biscuit} & Event driven & Dataflow & PCIe & Embedded \\
                Hadoop ISC* \cite{7524716} & Event driven & Dataflow (MapReduce) & SAS & Embedded \\
                YourSQL \cite{10.14778/2994509.2994512} & Query offloading (MySQL) & Declarative & PCIe (NVMe) & Bitsream\footnotemark[8] \\
                Summarizer \cite{10.1145/3123939.3124553} & Event driven & Shared memory & PCIe (NVMe) & Embedded \\
                NDP RE2 regex* \cite{10.1145/3211922.3211926} & Query offloading (Regex) & N.A. & N.A. & Embedded \\
                Registor \cite{10.1145/3310149} & Query offloading (Regex) & Shared memory & PCIe (NVMe) & Bitsream \\
                Cognitive SSD \cite{8839401} & Arbitrary code execution & Shared memory & PCIe (NVMe, OpenSSD) & Accelerators (Custom) \\
                INSIDER \cite{234968} & Event driven & Shared memory (VFS) & PCIe & Bitstream \\
                Catalina \cite{8855540} & Arbitrary code execution & Client / Server (MPI) & PCIe (NVMe) & Operating system (Linux) \\
                THRIFTY \cite{10.1145/3400302.3415723} & Event driven\footnotemark[9] & Shared memory (VFS)\footnotemark[9] & PCIe\footnotemark[9] & Bitstream\footnotemark[9] \\
                POLARDB \cite{246154} & Query offloading (POLARDB) & Declarative & PCIe & Bitstream \\
                NGD newport \cite{10.1145/3415580} & Arbitrary code execution & Client / Server & PCIe (NVMe) & Operating system (Linux) \\
                blockNDP \cite{10.1145/3429357.3430519} & Event driven & Dataflow (streams) & PCIe (NVMe, OpenSSD) & Virtual Machine (QEMU) \\
                QEMU CSD* \cite{10.1145/3439839.3459085} & Arbitrary code execution & Shared memory & PCIe (NVMe) & N.A. (Simulated) \\
                \bottomrule
            \end{tabular}
            \begin{tablenotes}[para,flushleft]
                \centering Overview of PFS works and various aspects as
                    previously detailed.
            \end{tablenotes}
        \end{threeparttable}
        \label{table:pfsoverview}
    \end{adjustbox}
\end{table}

\footnotetext[7]{Simulations performed by porting workloads unto ARM based
processor. No actual hardware on SSDs is used.}

\footnotetext[8]{The work uses special FCPs with hardware based filtering
functions. We assume these must be implemented using FPGAs although the work
does not specify.}

\footnotetext[9]{Build on top of the INSIDER software stack.}

% ---------------------------------------------------------------------------
% ----------------------- end of thesis sub-document ------------------------
% ---------------------------------------------------------------------------