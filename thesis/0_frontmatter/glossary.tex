% this file is called up by thesis.tex
% content in this file will be fed into the main document

% Glossary entries are defined with the command \nomenclature{1}{2}
% 1 = Entry name, e.g. abbreviation; 2 = Explanation
% You can place all explanations in this separate file or declare them in the middle of the text. Either way they will be collected in the glossary.

% required to print nomenclature name to page header
\markboth{\MakeUppercase{\nomname}}{\MakeUppercase{\nomname}}


% ----------------------- contents from here ------------------------


\nomenclature{HDD}{hard-disk drive} % 233
\nomenclature{SSD}{solid state drive} % 234
\nomenclature{Programmable Storage}{} % 246
\nomenclature{Computational Storage}{} % 246
\nomenclature{NVMe}{Non-Volatile Memory Express} % 250
\nomenclature{FTL}{Flash Translation Layer} % 255
\nomenclature{CSx}{Computational Storage Device} % 257
\nomenclature{ISA}{Instruction Set Architecture} % 274
\nomenclature{eBPF}{Extended Berkely Packet Filter} %293
\nomenclature{ZNS}{Zoned NameSpaces} % 294
\nomenclature{Host Managed}{} % 296
\nomenclature{Hybrid Filesystem}{} % 296
\nomenclature{LFS}{Log-Structured Filesystem} % 297
\nomenclature{FPGA}{Field-Programmable Gat Array} % 306
\nomenclature{PCIe}{Peripheral Component Interconnect Express} % 310
\nomenclature{OS}{Operating System} % 324
\nomenclature{VFS}{Virtual File System} % 326
\nomenclature{GPGPU}{General-Purpose computing on Graphics Processing Units} % 329
\nomenclature{API}{Application Programming Interface} % 330
\nomenclature{NAT}{Node Address Table} % 406
\nomenclature{CSEE}Computational Storage Execution Environments} % relatedwork 64
\nomenclature{RTL}Register-Transfer Level} % relatedwork 65

%\nomenclature{LSY}{ehbfuefebbfbjkjkebfjbfbfw} 
%\nomenclature{DEPC}{diethyl-pyro-carbonate; used to remove RNA-degrading enzymes (RNAases) from water and laboratory utensils}
%\nomenclature{DMSO}{dimethyl sulfoxide; organic solvent, readily passes through skin, cryoprotectant in cell culture}
%\nomenclature{EDTA}{Ethylene-diamine-tetraacetic acid; a chelating (two-pronged) molecule used to sequester most divalent (or trivalent) metal ions, such as calcium (Ca$^{2+}$) and magnesium (Mg$^{2+}$), copper (Cu$^{2+}$), or iron (Fe$^{2+}$ / Fe$^{3+}$)}
