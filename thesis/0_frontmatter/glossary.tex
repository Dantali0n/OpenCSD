% this file is called up by thesis.tex
% content in this file will be fed into the main document

% Glossary entries are defined with the command \nomenclature{1}{2}
% 1 = Entry name, e.g. abbreviation; 2 = Explanation
% You can place all explanations in this separate file or declare them in the middle of the text. Either way they will be collected in the glossary.

% required to print nomenclature name to page header
% \markboth{\MakeUppercase{\nomname}}{\MakeUppercase{\nomname}}


% ----------------------- contents from here ------------------------

\newlength{\nomenlabelindent}
\setlength{\nomenlabelindent}{13em}
\newenvironment{nomenclature2}{%
\newcommand\entry[2]{%
   \hangindent\nomenlabelindent\noindent\makebox[\nomenlabelindent][l]{##1\quad}\ignorespaces##2\par}%
   \addcontentsline{toc}{chapter}{\protect\numberline{}Glossary}%
   \section*{Glossary}}{\par\addvspace{12pt}}

\begin{nomenclature2}
    \entry{HDD}{Hard-Disk Drive} % 233
    \entry{SSD}{Solid-State Drive} % 234
    \entry{NVMe}{Non-Volatile Memory express} % 250
    \entry{FTL}{Flash Translation Layer} % 255
    \entry{CSx}{Computational Storage Device} % 257
    \entry{ISA}{Instruction Set Architecture} % 274
    \entry{eBPF}{extended Berkely Packet Filter} %293
    \entry{ZNS}{Zoned NameSpaces} % 294
    \entry{Host Managed}{Storage device that performs no automatic garbage
        collection or wear leveling exposing facilities to do so to the host} % 296
    \entry{Hybrid Filesystem}{Filesystem supporting both regular access as well
        as Computational Storage offloading} % 296
    \entry{LFS}{Log-Structured Filesystem} % 297
    \entry{FPGA}{Field-Programmable Gate Array} % 306
    \entry{PCIe}{Peripheral Component Interconnect express} % 310
    \entry{OS}{Operating System} % 324
    \entry{VFS}{Virtual FileSystem} % 326
    \entry{FUSE}{Filesystem in USErspace} % 327
    \entry{GPGPU}{General-Purpose computing on Graphics Processing Units} % introduction 149
    \entry{API}{Application Programming Interface} % introduction 150
    \entry{ABI}{Application Binary Interface} % introduction 159
    \entry{NAT}{Node Address Table} % 406
    \entry{OCSSD}{Open-Channel SSD} % relatedwork 29
    \entry{Programming Model}{See our previous survey
        \textit{Past, Present and Future of Computational Storage: A Survey} chapter 8.2 \cite{lukken2021past}} % relatedwork 48
    \entry{Degree of Programmability}{See our previous survey
        \textit{Past, Present and Future of Computational Storage: A Survey} chapter 8.2 \cite{lukken2021past}} % relatedwork 51
    \entry{RPC}{Remote Procedure Call} % relatedwork 57
    \entry{HTTP}{HyperText Transfer Protocol} % relatedwork 57
    \entry{SQL}{Structured Query Language} % relatedwork 96
    \entry{CSEE}{Computational Storage Execution Environment} % relatedwork 64
    \entry{SIT}{Segment Info Table} % relatedwork 64
    \entry{SSA}{Segment Summary Area} % relatedwork 64
    \entry{RTL}{Register-Transfer Level} % relatedwork 127
    \entry{NVMe-oF}{NVMe over Fabrics} % relatedwork 184
    \entry{MPI}{Message Passing Interface} % relatedwork 145
    \entry{HPC}{High-Performance Computing} % relatedwork 146
    \entry{CS}{Computer Science} % relatedwork 231
    \entry{POSIX}{Portable Operating System Interface} % design 101
    \entry{NIX}{A UNIX based operating system or derivative} % design 110
    \entry{PID}{Process IDentifier} % design 115
    \entry{Kernel}{Compiled binary or bytecode representation of a computational
        unit that can be submitted / scheduled for execution on accelerator type
        devices such as graphics cards or CSxs. Alternatively, resource and
        device management layer of an operating system.} % implementation 89
    \entry{EOF}{End Of File} % implementation 841
\end{nomenclature2}

% \printnomenclature

%\nomenclature{LSY}{ehbfuefebbfbjkjkebfjbfbfw} 
%\nomenclature{DEPC}{diethyl-pyro-carbonate; used to remove RNA-degrading enzymes (RNAases) from water and laboratory utensils}
%\nomenclature{DMSO}{dimethyl sulfoxide; organic solvent, readily passes through skin, cryoprotectant in cell culture}
%\nomenclature{EDTA}{Ethylene-diamine-tetraacetic acid; a chelating (two-pronged) molecule used to sequester most divalent (or trivalent) metal ions, such as calcium (Ca$^{2+}$) and magnesium (Mg$^{2+}$), copper (Cu$^{2+}$), or iron (Fe$^{2+}$ / Fe$^{3+}$)}
