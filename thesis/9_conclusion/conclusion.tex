\chapter{Conclusion}

To conclude, we have demonstrated that many previous prototype for flash based
CSxs exist and have been developed in the past decade. Until now, none of
these prototype where able to provide concurrent regular and offloaded
filesystem access, aptly named \textit{hybrid filesystems}. Our simulation
framework \textit{OpenCSD} and filesystem \textit{FluffleFS} are able to perform
offloading entirely using existing APIs, no shared libraries or network
communication is necessary. In addition, our kernels are vendor agnostic and
filesystem aware. Finally, our prototype runs entirely in userspace requiring no
understanding of the Linux kernel or modifications.

To do this our offloading is triggered through filesystem extended attributes
resulting in complete isolation from regular operating system behavior. These
filesystem extended attributes are present in all modern operating systems
including Windows, MacOs, FreeBSD and Linux. Across all previous CSx works only
a single other work managed to use existing APIs and in doing so they changed
UNIX pipe behavior \cite{10.1145/3342195.3387557}.

Concurrent regular and offloaded access is achieved by utilizing snapshot
consistency with LFS and ZNS technologies. Furthermore,
by utilizing eBPF and uBPF these kernels written for offloading can be made
vendor and filesystem agnostic. While the API for kernel filesystem support
still needs to be formalized and the proposed CSx runtime is still absent in our
prototype, the possibility of filesystem awareness in a concurrent regular and
offloaded setting has been proven. Furthermore we have proven that regular and
offloaded accesses can be easily differentiated utilized PID inode pairs.

Throughout this work we have argued why LFS, ZNS, eBPF and extended attributes
are the current best suited technologies to implement a hybrid filesystem.

We believe that many CSx prototype suffer from a high barrier to entree due to
complicated setup and technologies. Therefor, our design focussed heavily on
being easy to use to reduce barrier to entree. Our design has achieved this by
only utilizing userspace technologies, requiring no knowledge of kernel
development or any kernel modifications. In addition, our framework
\textit{OpenCSD} utilizes a component architecture that easily allows for
replacement of used technologies.

During the evaluation we showed that offloaded applications can reduce
data movement between the device and host by 99.9\% as well as drastically
reduce the host CPU load. However, the overall performance of FluffleFS still
needs to be improved significantly, only able to perform on par with flash
optimized filesystems under very select circumstances. Furthermore, our
limited performance optimizations made us unable to proof that concurrent
offloaded and regular file access can achieve unstagnated performance.

We encourage everyone to try our prototype today which is readily available
and open-source published under a permissive license \cite{qemu-csd}. Along our
source code are datasets of our measurements as well as the scripts to perform
these measurements and generate accompanying graphs. Even the source files for
this very thesis are included. We hope this openness inspires others to do the
same.

% ---------------------------------------------------------------------------
% ----------------------- end of thesis sub-document ------------------------
% ---------------------------------------------------------------------------