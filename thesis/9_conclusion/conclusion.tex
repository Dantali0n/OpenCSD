\chapter{Conclusion}

To conclude, we have demonstrated that many previous prototype for flash based
CSxs exist and have been developed in the past decade. Until now, none of
these prototype where able to provide concurrent regular and offloaded
filesystem access (R1), aptly named \textit{hybrid filesystems}. Our emulation
framework \textit{OpenCSD} and filesystem \textit{FluffleFS} are able to perform
offloading entirely using existing APIs (RD4), no shared libraries or network
communication is necessary. In addition, our kernels are vendor agnostic and
filesystem aware (RD6).

\subsubsection*{(RD1) Best suited existing technologies for a hybrid LFS}

Throughout this work we have argued why LFS, ZNS, eBPF and extended attributes
are the current best suited technologies to implement a hybrid filesystem.
With the ability to reuse existing operating systems APIs, provide
a snaphshot consistency model, reduce translations between host and device
(semantic gap) and provide vendor agnostic kernels among the core reasons.

It is through these technologies and by utilizing a snapshot
consistency model that concurrent regular and offloaded can be achieved (R1).

\subsubsection*{(RD2) Minimizing complexity for a hybrid LFS}

Using these technologies our prototype runs entirely in userspace requiring no
understanding of the Linux kernel or modifications. In addition, interacting
with offloading can be done in minimal lines of code using common system call
such as \textit{open}, \textit{read} and \textit{setxattr}. The use of eBPF
further reduces complexity as development can be done using the common C
programming language.

\subsubsection*{(RD3) Simple replacement of existing technologies}

Our framework \textit{OpenCSD} utilizes a component architure in addition to
separating frontend and backends with interfaces. this allows for simple
replacement of any used technology.

\subsubsection*{(RD4) Registering CSx compute kernels using existing operating
system APIs}

Our offloading is triggered through filesystem extended attributes
resulting in complete isolation from regular operating system behavior.
These filesystem extended attributes are present in all modern operating systems
including Windows, MacOs, FreeBSD and Linux. Across all previous CSx works only
a single other work managed to use existing APIs and in doing so they changed
UNIX pipe behavior \cite{10.1145/3342195.3387557}.

\subsubsection*{(RD5) Differentiating individual users, files and I/O operations in
relation to their CSx compute kernel}

We have proven that regular and offloaded accesses can be easily differentiated
utlizing PID inode pairs.

\subsubsection*{(RD6) Reusing the compute kernel API across devices}

By utilizing eBPF and uBPF these kernels written for offloading can be made
vendor and filesystem agnostic. While the API for kernel filesystem support
still needs to be formalized and the proposed CSx runtime is still absent in our
prototype, the possibility of filesystem awareness in a concurrent regular and
offloaded setting has been proven.

\subsubsection*{(RD7) Ensure user submitted kernels are safe}

Throughout our consideration section we have defined three concrete
methodologies that can be used to ensure safety of user submitted kernels.
In addition, as inline with our other research questions most recommendations
utilize existing technologies.

\subsubsection*{(RE1) Undisturbed performance for concurrent regular and
offloaded file access in a hybrid LFS}

Our limited performance optimizations made us unable to proof that concurrent
offloaded and regular file access can achieve undisturbed performance.

\subsubsection*{(RE2, RE3) Reduced host CPU load and data movement for
asynchronous offloaded applications}

During the evaluation we showed that offloaded applications can reduce
data movement between the device and host by 99.9\% as well as drastically
reduce the host CPU load. However, the overall performance of FluffleFS still
needs to be improved significantly, only able to perform on par with flash
optimized filesystems under very select circumstances. 

\subsection*{Published under open science principles}

We encourage everyone to try our prototype today which is readily available
and open-source published under a permissive license \cite{qemu-csd}. Along our
source code are datasets of our measurements as well as the scripts to perform
these measurements and generate accompanying graphs. Even the source files for
this very thesis are included. We hope this openness inspires others to do the
same.

% ---------------------------------------------------------------------------
% ----------------------- end of thesis sub-document ------------------------
% ---------------------------------------------------------------------------