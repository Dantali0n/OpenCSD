
% this file is called up by thesis.tex
% content in this file will be fed into the main document

%: ----------------------- introduction file header -----------------------
\chapter{Introduction}

% the code below specifies where the figures are stored
\ifpdf
    \graphicspath{{1_introduction/figures/PNG/}{1_introduction/figures/PDF/}{1_introduction/figures/}}
\else
    \graphicspath{{1_introduction/figures/EPS/}{1_introduction/figures/}}
\fi

% ----------------------------------------------------------------------
%: ----------------------- introduction content ----------------------- 
% ----------------------------------------------------------------------



%: ----------------------- HELP: latex document organisation
% the commands below help you to subdivide and organise your thesis
%    \chapter{}       = level 1, top level
%    \section{}       = level 2
%    \subsection{}    = level 3
%    \subsubsection{} = level 4
% note that everything after the percentage sign is hidden from output



\section{Context} 

% We have a separate previous work section we will ignore most of the different
% related literature in the introduction.

% Problem

The amount of data being generated and processed worldwide is growing rapidly.
Meanwhile the prevelant Von Nuemann computer architecture requires all data is 
moved to system memory before it can be processed \cite{2018-neumann-bottleneck}.
Until recently storage devices such as hard-disk drives (HDDs) or solid state
drives (SSDs) have only been passively storing data. With the increasing
bandwidth persistent storage devices like SSDs achieve this excessive data
movement to memory is increasingly becoming a bottleneck \cite{2014-micro-ndp}.
Similarly, the CPU generational performance
improvements \cite{2016-western-digital} as well as link speeds of interconnects
are stagnating compared to nand flash bandwidth \cite{10.1145/3286588}, the
underlying technology enabling modern SSDs.

% Potential solution and benefits

A promising solution to this problem is the processing of data in-situ by
pushing compute to the storage layer. This solution is actively being researched
as \textit{"Programmable Storage"} and \textit{"Computational Storage"} among
other terms. The concept is not new however, as it has been previously
explored in database mainframes \cite{database-computer} as well as for
conventional HDDs \cite{active-disk-pillar, active-disks-tech,
intelligent-disk}. With rise of Non-Volatile Memory Express (NVMe) technologies
offering signifcantly more device-internal bandwidth this concept is being
revisted. What makes NVMe SSDs even more suited is that they already contain
capable processing units often boasting even multiple cores. This requirement
comes from the internal management the device has to perform known as the
\textit{Flash Translation Layer} (FTL). Devices utilizing computational elements
on conventional SSDs for \textit{Computational Storage} are commonly known as
\textit{Computational Storage Devices} (CSx)s. The potential benefits of such a
heteregenous architecture offering programmable storage include energy saving,
cost reduction and performance improvements.

% Challenges

Despite all these benefits there is still no widespread adoption even after a
decade of research \cite{lukken2021past}. A multitude of challenges
have prohibited this adoption of which several still applicable today. Four
prominent challenges include that, firstly, \textit{Computational Storage}
requires complete vertical integration. Meaning that changes are required at all
levels of abstraction, device hardware, interfaces, drivers and operating system
to name a few. Trying to integrate a solution for all levels in one prototype
results in a very large problem space. This large problem space complicates
deriving standards for designs and interfaces although the development of such a
standard by SNIA is on-going \cite{snia-model}. Secondly, vendors might choose
to use different hardware with different
\textit{Instruction Set Architectures} (ISA)s or different host-to-device
interfaces. These differences might result in incompatible user applications
across vendors hurting reusability and hindering adoption. Third, filesystems
are managed by the host operating system while the FTL is
managed by the device. Given that one is not aware of the processes within the
other, this semantic gap complicates several aspects such as consistency,
concurreny, multi-user tenancy and filesystem integration. Lastly, no
specialized filesystem designs exist that support both regular user access
concurrent with \textit{Computational Storage} offloading.

% Solutions

In this work we address each of these challenges directly and propose a
complete \textit{Computational Storage} solution that offers a filesystem
capable of concurrent multi-user tenancy with both regular and compute
offloading access (hybrid). We introduce each of the solutions briefly before
describing their complete case. Firstly, we circumvent the complexity of
complete vertical integration by creating a simulation platform on top of
QEMU \cite{qemu}. Secondly, We eliminate potential incompatibities across vendors
by using \textit{Extended Berkely Packet Filer} (eBPF) \cite{what-ebpf} as
compute kernel language. Third, We bridge the semantic gap by using Zoned
Namespaces (ZNS) \cite{zns} SSDs which moves the FTL to the host operating
system (host-managed). Lastly, we allow for concurrent multi-user tenancy in
a \textit{Hybrid Filesystem} context by developing a specialized
\textit{Log-Structured Filesystem} (LFS) \cite{Rosenblum1992TheDA} with an
in-memory snapshot consistency model \cite{Viotti2016ConsistencyIN}. We present
this complete solution as \textit{OpenCSD} an open-source Apache 2 licensed CSx
simulation framework with accompanying filesystem called
FluffleFS \cite{qemu-csd}.

\subsection*{A Case for CSx Simulation}

The lack of a standardized design has lead to a large variety of different
hardware architectures being explored including using embedded CPUs or
\textit{Field-Programmable Gate Arrays} (FPGA)s. Moreover, closed-source
devices have even become commercially available. Still there is a lot of
uncertainty about the right computational hardware models, interconnects and
interconnects their interfaces. Even though the \textit{Peripheral Component
Interconnect Express} (PCIe) interconnect and the NVMe storage interface
currently dominate modern storage devices it is unclear if their current
capabilities are sufficient to support CSxs. Meanwhile existing technologies
such as QEMU allow rapid development of new simulated hardware.

The lack of a clear standardized design combined with the ease of prototyping
designs in simulation clearly shows that choosing simulation over an actual
hardware prototype is the current logical choice.

\subsection*{A Case for eBPF Programmability}

The concept of programmability provides end users the capability to run their
own provided code. With the introduction of modern \textit{Operating Systems}
(OS)s this is expected to happen in safe and dynamic manner. Programmability
can be achieved in different manners such as through the kernel in kernel
modules, with filesystems through \textit{Virtual File System} (VFS) or FUSE and
with language runtimes such as those in Python. In addition programmability can
also target a peripheral instead of the host directly such as is prevelant in
\textit{General-Purpose computing on Graphics Processing Units} (GPGPU)
programming through \textit{Application Programming Interfaces} (API)s like
OpenCL \cite{opencl} and CUDA \cite{cuda}. Through similar mechanisms
programmability in CSxs can be achieved.

The uncertainty of the hardware design makes it unclear exactly how close to the
hardware and through which method of programmability CSxs will be effective.
Although, naturally, the closer to the actual storage the better
(Near-Data Processing). Therefor, our programmability must not be restricted or
favor a particular type.

Introducing eBPF an ISA and API with a large collection of toolchains and
wideranging implementations \cite{what-ebpf, McCanne1993TheBP}. These
implementations range from complex such as the one found in the Linux kernel to
simple such as the one found in uBPF \cite{ubpf}, with the key difference in
complexity being the supported API. Effectively, the implementation running
(runtime) eBPF code decides the API by tying special eBPF syscall instructions
to predefined code. In addition, this allows the user program and runtime to
exchange important information such as prominently done in Linux through eBPF
maps \cite{bpf-man}.

The benefits of eBPF are four fold. First, the ISA is not tailored to any
specific domain and it has been used in networking \cite{xdp},
tracing \cite{enhanced-ebpf} and security \cite{seccomp} applications. Second,
the simple nature of the eBPF ISA allows for verification and bounded execution
checking such as performed by the Linux kernel \cite{kern-analysis}. Third, eBPF
supports efficient code generation through jitting achieving close to bare-metal
performance. Lastly, eBPF has been positioned as the unified ISA for
heteregenous computing \cite{Brunella2020hXDPES, bpf-uapi}.

\subsection*{A Case for ZNS}

A new emerging NVMe standard is Zoned Namedspaces (ZNS). It can been seen as the
technical successor to Open-Channel % \cite{}.
This standard allows host visibility and control over data placement
(host-managed) while more closely representing nand flash behavior. This
replacement for the traditional block interface offers reduced
write-ampplifcation, lower SSD hardware requirements and more intelligent
wear-leveling and garbage collection. However ZNS SSDs come with several
constraints such as not allowing in-place updates (append-only), using a large
collection of sectors as single erasure unit and requiring wear-leveling and
garbage collection to be explicitly programmed by the host.

Yet we argue ZNS greatly improves the feasibility of CSx supporting
\textit{hybrid filesystems}. Firstly, ZNS offers more predictable performance in
conjuction with \textit{Log-Structured Filesystems} as the absence of an FTL
means the behavior of \textit{append-only} writes won't be influenced by
underlying write-amplification or garbage-collection. Secondly, the direct
relationship between dimensions as reported to the host and to those known on
the device results in a greatly simplified exchange of information from host
submitted compute kernels as well as reduced kernel runtime translations.
Finally, this reduced semantic gap between the host and device is essential to
compute kernels running autonomously and without shared virtual memory.

\subsection*{A Case for LFS}

\textit{Log-Structured Filesystems} have been around for a long
time \cite{Rosenblum1992TheDA} but have not really been popularized until the
recent advancement of nand flash. A LFS maintains one or multiple logs which
are append-only sections of the filesystem. This has the advantage of the
underlying device receiving filesystem writes as sequential I/O operations,
which is known to improve performance. In addition, LFSs are often relatively
easy to implement. Unfortunately, LFSs suffer from write-amplification because
modifcations updating inode or other data location information travels up the
chain in history.

Yet we argue an LFS is essential for an effective \textit{hybrid filesystem} due
to the following three reasons. Firstly, the append-only nature of a LFS is the
best fit for the append-only requirement of ZNS SSDs. Secondly, the
chronological order of logs allow for snapshotting, versioning and simplified
crash recovery. It is thanks to the properties of an LFS that FluffleFS is able
to provide in-memory snapshots to compute kernels. Lastly, the problem around
write-amplification is a solved issue thanks to the F2FS \cite{Lee2015F2FSAN}
work which introduced so called Node Address Tables (NAT).

\section{Research Question}

How to achieve multi-user tenancy in a Log Structured Filesystem (LFS) while
supporting Computational Storage offloading?

\subsection{Sub Questions}

What would a Zoned Namespaces (ZNS) Log Structured Filesystem (LFS) look like?

How to register Computational Storage Device (CSx) compute kernels using
existing operating system APIs?

How to differentiate individual users, files and I/O operations in relation to
their CSx compute kernel?

\section{Research Method}

%  Iterative design approach
