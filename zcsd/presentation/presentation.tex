\documentclass{beamer}
\usepackage[utf8]{inputenc}
\usepackage[inline]{enumitem}
\usepackage{pdfpages}
\usepackage{listings}
\usepackage{subcaption}
\usepackage{wasysym}
\usepackage[export]{adjustbox}[2011/08/13]
\usetheme{cern}

\newcommand{\btVFill}{\vskip0pt plus 1filll}

% fix tilde
% \newcommand{\propertilde}{\raise.17ex\hbox{$\scriptstyle\mathtt{\sim}$}}

% \setbeameroption{show notes on second screen=right}

%The CERN logo is legally protected. Please visit http://cern.ch/copyright for
%information on the terms of use of CERN content, including the CERN logo.

% The optional `\author` command defines the author and is displayed in the
%slide produced by the `\titlepage` command.
\author{Corné Kenneth Lukken}

% The optional `\title` command defines the title and is displayed in the slide
%produced by the `\titlepage` command.
\title{OpenChannelSSD vs ZNS SSD}

% The optional `\subtitle` command will add a smaller title below the main one,
%and will not be displayed in any of the slides' footer.
\subtitle{A SPDK compared to lightnvm story}

% The optional `\date` command will display a custom free text date on the all
%of the slides' footer. If omitted today's date will be used.
%\date{Monday, 1st January 2018}

% ------------------------------------------------------------------------%
% Proper Python Syntax Highlighting                                       %
% Author: redmode
% https://tex.stackexchange.com/questions/83882/how-to-highlight-python   %
% -syntax-in-latex-listings-lstinputlistings-command#83883                %
% ----------------------------------------------------------------------- %

% Default fixed font does not support bold face
\DeclareFixedFont{\ttb}{T1}{txtt}{bx}{n}{6} % for bold
\DeclareFixedFont{\ttm}{T1}{txtt}{m}{n}{6}  % for normal

% Custom colors
\definecolor{deepblue}{rgb}{0,0,0.5}
\definecolor{deepred}{rgb}{0.6,0,0}
\definecolor{deepgreen}{rgb}{0,0.5,0}

% Python style for highlighting
\newcommand\pythonstyle{
	\lstset{
		language=Python,
		basicstyle=\ttm,
		showstringspaces=false,
		tabsize=2,
		aboveskip=0.1cm,
		belowskip=0.1cm,
		otherkeywords={self},             % Add keywords here
		keywordstyle=\ttb\color{deepblue},
		emph={MyClass,__init__},          % Custom highlighting
		emphstyle=\ttb\color{deepred},    % Custom highlighting style
		stringstyle=\color{deepgreen},
		frame=tb,                          % Any extra options here
		prebreak=\textbackslash,
		linewidth=7.00cm,
		breaklines=true,
	}
}

% Python environment
\lstnewenvironment{python}[1][] {
	\pythonstyle\lstset{#1}
}{}

% Python for inline
\newcommand\pythoninline[1]{{\pythonstyle\lstinline!#1!}}

% Python for external file
\newcommand\pythonexternal[2][]{{\pythonstyle\lstinputlisting[#1]{#2}}}

\newcommand\pythonfullstyle{
	\lstset{
		language=Python,
		basicstyle=\ttm,
		showstringspaces=false,
		tabsize=2,
		aboveskip=0.1cm,
		belowskip=0.1cm,
		otherkeywords={self},             % Add keywords here
		keywordstyle=\ttb\color{deepblue},
		emph={MyClass,__init__},          % Custom highlighting
		emphstyle=\ttb\color{deepred},    % Custom highlighting style
		stringstyle=\color{deepgreen},
		frame=tb,                          % Any extra options here
		prebreak=\textbackslash,
		linewidth=11.00cm,
		breaklines=true,
	}
}

% Python environment
\lstnewenvironment{pythonfull}[1][] {
	\pythonfullstyle\lstset{#1}
}{}

% Python for inline
\newcommand\pythonfullinline[1]{{\pythonfullstyle\lstinline!#1!}}

% Python for external file
\newcommand\pythonfullexternal[2][]{{\pythonfullstyle\lstinputlisting[#1]{#2}}}

% ----------------------------------------------------------------------- %

% Bash style for highlighting
\newcommand\bashstyle{
	\lstset{
		language=Bash,
		basicstyle=\ttm,
		showstringspaces=false,
		tabsize=2,
		%commentstyle=itshape,
		aboveskip=0.1cm,
		belowskip=0.1cm,
		prebreak=\textbackslash,
		extendedchars=true,
		mathescape=false,
		% literate= {\$}{{\textcolor{blue}{\$}}}1 {&}{{\textcolor{blue}{\&}}}1 {/n}{{\textcolor{green}{\textbackslash n}}}1,
		linewidth=11cm,
		breaklines=true
	}
}

% Bash environment
\lstnewenvironment{bash}[1][] {
	\bashstyle\lstset{#1}
}{}

% Bash for inline
\newcommand\bashinline[1]{{\bashstyle\lstinline!#1!}}

% Bash for external file
\newcommand\bashexternal[2][]{{\bashstyle\lstinputlisting[#1]{#2}}}


\begin{document}

% \frontcover

% The optional `\titlepage` command will create a slide with the presentation's
%title, subtitle and author.
\frame{\titlepage}

% The optional `\tableofcontents` command will automatically create a table of
%contents based pm the sections.
% \frame{\tableofcontents}

% \section{Introduction}

\begin{frame}{OpenChannelSSD Recap}
	\begingroup
	\small
%	\begin{columns}
		\begin{itemize}
			\item Exposed internal structure, Groups, PUs, Chunks, Sectors.
			\item Implicit chunk state machine.
			\item Nice direct mapping between LBA \& structure
				  (assuming values are clean powers of two).
		\end{itemize}
%	\end{columns}
	\endgroup
\end{frame}

\begin{frame}{ZNS SSD Recap}
	\begingroup
	\small
%	\begin{columns}
		\begin{itemize}
			\item Internal structure not clearly exposed, namespaces, zones, LBAs.
			\item Zone state implicitly managed, querying current zone states easy.
			\item Limit on the number of commands that can be in
				  progress across zones (active zones).
			\item Limit on the number of zones that can be in state
				  opened (open zones).
		\end{itemize}
%	\end{columns}
	\endgroup
\end{frame}

\begin{frame}{Comparison off specifications}
	\begingroup
	\small
	\begin{itemize}
		\item Zone state model \& Chunk state model very similar.
		\item Zone append provides no guarantees on which write ends up at
			  which LBA unlike OCSSD vectored write.
		\item Zoned read, write, flush, etc are part of regular NVMe
			  specification with slight extension to support zones. While OCSSD
			  is entirely new specification.
		\item ZNS SSD expects user to retrieve LBAs of zone append
			  asynchronously.
	\end{itemize}
	\endgroup
\end{frame}

\begin{frame}{Differences in abstractions}
	\begingroup
	\small
	\begin{itemize}
		\item SPDK has no regular write / read commands specific to
			  zones.
		\item lightnvm supports using internal structure when performing
			  commands instead of LBA. Support functions \textit{addr\_gen2dev}
			  \textit{addr\_dev2gen} breakdown if any value for internal
			  structure is not clean power of two! The anonymous union in
			  \textit{struct nvm\_addr} should handle this?
		\item Identifying zone boundaries is much easier compared to OCSSD
			  structure boundaries.
		\item lightnvm is blocking, SPDK uses lockless I/O queues and callbacks.
	\end{itemize}
	\endgroup
\end{frame}

\end{document}