\documentclass[conference]{IEEEtran}
\IEEEoverridecommandlockouts
% The preceding line is only needed to identify funding in the first footnote.
% If that is unneeded, please comment it out.
%\usepackage[backend=bibtex,style=verbose-trad2]{biblatex}

\usepackage{adjustbox}
\usepackage{algorithmic}
\usepackage{amsmath,amssymb,amsfonts}
%\usepackage[backend=bibtex,style=ieee]{biblatex}
%\usepackage{bookmark}
\usepackage{booktabs}
\usepackage{caption}
\usepackage{cite}
\usepackage{color}
\usepackage[inline]{enumitem}
\usepackage{float}
\usepackage[T1]{fontenc}
%\usepackage{fontspec}
\usepackage{footnote}
\usepackage{graphicx}
\usepackage[colorlinks=true,citecolor=blue]{hyperref}
\usepackage{inputenc}[utf8]
\usepackage{listings}
\usepackage{textcomp}
\usepackage[flushleft]{threeparttable}
\usepackage{subcaption}
\usepackage{xcolor}

\title{QEMU Computation Storage Devices%\\
%{\footnotesize \textsuperscript{*}Note: Sub-titles are not captured in Xplore 
% and should not be used}
%\thanks{Identify applicable funding agency here. If none, delete this.}
}

\pagenumbering{arabic}
\pagestyle{plain}

% Ensure decimal numbering in table  of contents
\renewcommand{\thesection}{\arabic{section}}
\renewcommand{\thesubsection}{\arabic{section}.\arabic{subsection}}
\renewcommand{\thesubsubsection}{\arabic{section}.\arabic{subsection}.\arabic{subsubsection}}

% ensure decimal numbering for sub sections
\makeatletter
\renewcommand{\@seccntformat}[1]{\csname the#1\endcsname.\quad}
\makeatother

% ------------------------------------------------------------------------%
% Proper Python Syntax Highlighting                                       %
% Author: redmode
% https://tex.stackexchange.com/questions/83882/how-to-highlight-python   %
% -syntax-in-latex-listings-lstinputlistings-command#83883                %
% ----------------------------------------------------------------------- %

% Default fixed font does not support bold face
\DeclareFixedFont{\ttb}{T1}{txtt}{bx}{n}{8} % for bold
\DeclareFixedFont{\ttm}{T1}{txtt}{m}{n}{8}  % for normal

% Custom colors
\definecolor{deepblue}{rgb}{0,0,0.5}
\definecolor{deepred}{rgb}{0.6,0,0}
\definecolor{deepgreen}{rgb}{0,0.5,0}

% Python style for highlighting
\newcommand\pythonstyle{
	\lstset{
		language=Python,
		basicstyle=\ttm,
		showstringspaces=false,
		tabsize=4,
		aboveskip=0.2cm,
		belowskip=0.2cm,
		otherkeywords={self},             % Add keywords here
		keywordstyle=\ttb\color{deepblue},
		emph={MyClass,__init__},          % Custom highlighting
		emphstyle=\ttb\color{deepred},    % Custom highlighting style
		stringstyle=\color{deepgreen},
		frame=tb,                          % Any extra options here
		prebreak=\textbackslash,
		linewidth=8.85cm,
		breaklines=true,
	}
}

% Python environment
\lstnewenvironment{python}[1][] {
	\pythonstyle\lstset{#1}
}{}

% Python for inline
\newcommand\pythoninline[1]{{\pythonstyle\lstinline!#1!}}

% Python for external file
\newcommand\pythonexternal[2][]{{\pythonstyle\lstinputlisting[#1]{#2}}}

% ----------------------------------------------------------------------- %

% Bash style for highlighting
\newcommand\bashstyle{
	\lstset{
		language=Bash,
		basicstyle=\ttm,
		showstringspaces=false,
		tabsize=2,
		%commentstyle=itshape,
		aboveskip=0.2cm,
		belowskip=0.2cm,
		prebreak=\textbackslash,
		extendedchars=true,
		mathescape=false,
		% literate= {\$}{{\textcolor{blue}{\$}}}1 {&}{{\textcolor{blue}{\&}}}1 {/n}{{\textcolor{green}{\textbackslash n}}}1,
		linewidth=8.85cm,
		breaklines=true
	}
}

% Bash environment
\lstnewenvironment{bash}[1][] {
	\bashstyle\lstset{#1}
}{}

% Bash for inline
\newcommand\bashinline[1]{{\bashstyle\lstinline!#1!}}

% Bash for external file
\newcommand\bashexternal[2][]{{\bashstyle\lstinputlisting[#1]{#2}}}


% Python style for highlighting
\newcommand\cstyle{
	\lstset{
		language=c,
		basicstyle=\ttm,
		showstringspaces=false,
		tabsize=4,
		aboveskip=0.2cm,
		belowskip=0.2cm,
		otherkeywords={self},             % Add keywords here
		keywordstyle=\ttb\color{deepblue},
		emph={MyClass,__init__},          % Custom highlighting
		emphstyle=\ttb\color{deepred},    % Custom highlighting style
		stringstyle=\color{deepgreen},
		frame=tb,                          % Any extra options here
		prebreak=\textbackslash,
		linewidth=8.85cm,
		breaklines=true,
	}
}

% Python environment
\lstnewenvironment{clist}[1][] {
	\cstyle\lstset{#1}
}{}

% Python for inline
\newcommand\cinline[1]{{\cstyle\lstinline!#1!}}

% Python for external file
\newcommand\cexternal[2][]{{\cstyle\lstinputlisting[#1]{#2}}}

% ----------------------------------------------------------------------- %

\begin{document}

\begin{titlepage}
\begingroup
\centering
{\LARGE\bfseries QEMU Computational Storage Devices}

\vspace{1cm}

{\Large Vrije Universiteit (VU)}

\vspace{0.5cm}

{Corne Kenneth Lukken}

{\textit{Department of Computer Science} \\
Amsterdam, Netherlands \\
info@dantalion.nl}

\vspace{4.0cm}

%\begin{figure}[H]
%	\centering
%	\includegraphics[width=0.7\textwidth]{resources/images/cover-image.png}
%	\captionsetup{justification=centering}
%	\caption{This is the cover image}
%\end{figure}

\vfill
\endgroup
\hfill
\begin{minipage}{0.3\textwidth}
\begin{flushright}
	\includegraphics[width=\textwidth]{resources/images/vu-logo.png}
\end{flushright}
\end{minipage}
\end{titlepage}

\clearpage
\onecolumn

% Ensure black link color in table of contents
\hypersetup{
	linkcolor=black
}

\renewcommand{\contentsname}{CONTENTS}
\tableofcontents{}

\hypersetup{
	linkcolor=blue
}

\twocolumn

\addcontentsline{toc}{section}{\protect\numberline{}INTRODUCTION}
\section*{INTRODUCTION}
This is the introduction

%\footnotemark[1]. example\ref{term}.
%\footnotetext[1]{test}
%\subsection*{Accelerated Computing Landscape}
%\textit{C++}
%GPUOCelot\cite{tired-manycore-architectures-ocelot}
%\section{LARGER SECTION}
%$\mathcal{O}(N\log N)$

%\begin{center}
%	\begin{figure}[H]
%		\includegraphics[width=0.5\textwidth]{resources/images/example.png}
%		\captionsetup{justification=centering}
%		\caption{
%			\href{https://commons.wikimedia.org/wiki/User:AkanoToE}{AkanoToE}
%			licensed under CC-BY-SA 4.0}
%		\label{fig:consampling}
%	\end{figure}
%\end{center}

%\begin{table}[h!]
%	\caption{Input evaluation sizes}
%	\label{table:files}
%	\centering
%	\begin{adjustbox}{width=0.5\textwidth}
%		\begin{threeparttable}[]
%			\begin{tabular}{llll}
%				\toprule
%				\textbf{Name} & \textbf{Type} & \textbf{Sample rate (kHz)} &
%				\textbf{Samples} \\
%				\midrule
%				test-b & complex & 10000 & 502622 \\
%				test-t & complex & 5333 & 2654207 \\
%				\bottomrule
%			\end{tabular}
%			\begin{tablenotes}[para,flushleft]
%				\centering List of input files and parameters used for
%				performance evaluations.
%			\end{tablenotes}
%		\end{threeparttable}
%	\end{adjustbox}
%\end{table}

\section{Target Hardware}

\begin{center}
	\begin{figure}[H]
		\bashexternal{resources/bash/inxi.sh}
		\captionsetup{justification=centering}
		\caption{Output of inxi showing both CPU and GPU hardware configuration}
		\label{fig:inxihardware}
	\end{figure}
\end{center}

%\cite{Rius1995,Karp1996,parreverse,Adikaram2014}.

%\begin{center}
%	\begin{figure}[H]
%		\cexternal{resources/c/basic.c}
%		\captionsetup{justification=centering}
%		\caption{First basic kernel}
%		\label{fig:base}
%	\end{figure}
%\end{center}

\addcontentsline{toc}{section}{\protect\numberline{}TERMINOLOGY} 
\section*{TERMINOLOGY} \label{term}

This section hopes to define some common terminology as well as clearly detail
how some of these terms are interpreted. Mainly, this section serves to avoid
confusion.

\begin{enumerate}
	\item CSD - Computational Storage Device
\end{enumerate}

% \phantomsection
% \addcontentsline{toc}{section}{\protect\numberline{}References}
%\bibliographystyle{IEEEtran}
%\bibliography{bibliography}


%% \begin{center}
% 	\begin{figure}[H]
% 		\includegraphics[width=0.5\textwidth]{resources/mape-k.png}
% 		\captionsetup{justification=centering}
% 		\caption{Diagram of MAPE-K feedback loop \\ CC-BY 4.0$^{0}$}
% 		\label{fig:mapek}
% 	\end{figure}
% \end{center}

% \begin{table}[h!]
% \caption{Platform migrations}
% \label{table:platformmigrations}
% \centering
% \begin{adjustbox}{width=0.5\textwidth}
% \begin{threeparttable}[]
% \begin{tabular}{lll}
% \toprule 
% \textbf{Original platform} & \textbf{New platform} & \textbf{Status} \\
% \midrule
% \href{https://git.openstack.org}{git.openstack.org} &
% \href{https://opendev.org}{opendev.org} & complete \\
% \href{https://review.openstack.org}{review.openstack.org} &
% \href{https://review.opendev.org}{review.opendev.org} & complete \\
% \href{https://launchpad.net}{launchpad.net} &
% \href{https://storyboard.openstack.org/}{storyboard.openstack.org} & ongoing \\
% \bottomrule
% \end{tabular}
% \begin{tablenotes}[para,flushleft]
%      \centering List of platform migrations and their status.
% \end{tablenotes}
% \end{threeparttable}
% \end{adjustbox}
% \end{table}

% \addcontentsline{toc}{section}{\protect\numberline{}Abbreviations}
% \section*{Abbreviations}

% \begin{enumerate}
% 	\item AUAS  - Amsterdam University of Applied Sciences
% 	\item TI    - Technical Informatics
% 	\item ALICE - A Large Ion Collider Experiment
% 	\item RPS   - Resource Provisioning Services
% 	\item API   - Application Programming Interface
% 	\item AQMP  - Advanced Message Queuing Protocol
% 	\item UUID  - Universally unique identifier
% 	\item R\&D  - Research and Development
% 	\item IaaS  - Infrastructure as a Service
% 	\item CERN  - European Organization for Nuclear Research
% 	\item LHC   - Large Hadron Collider
% 	\item PTL   - Project Team Lead(er)
% 	\item REST  - REpresentational State Transfer
% 	\item RFC   - Request For Comments
% 	\item HTTP  - HyperText Transfer Protocol
% 	\item SQL   - Structured Query Language
% 	\item YAML  - YAML Ain't Markup Language
% 	\item EOL   - End Of Life
% 	\item OOM   - Out Of Memory
% 	\item JSON  - JavaScript Object Notation
% 	\item PoC   - Proof of Concept
% 	\item CU    - Compute Unit
% 	\item RAM   - Random Access Memory
% \end{enumerate}

% \onecolumn

% \addcontentsline{toc}{section}{\protect\numberline{}Appendix}
% \section*{\large Appendix}

% \begin{table}[h!]
% \addcontentsline{toc}{subsection}{\protect\numberline{}Watcher collaboration
% resources}
% \caption{Online resources regarding governance \& collaboration for Watcher}
% \label{table:collabresources}
% \centering
% \begin{adjustbox}{width=\textwidth}
% \begin{threeparttable}[]
% \begin{tabular}{ll}
% \toprule 
% \textbf{Source} &\textbf{Content} \\
% \midrule
% https://www.openstack.org/ & Entry into many important other online resources \\
% https://wiki.openstack.org/wiki/Watcher & Watcher entry for many online 
% resources \\
% https://launchpad.net/watcher & Bug tracker for issues of OpenStack components
% such as Watcher \\
% https://storyboard.openstack.org/ & New bug tracker to eventual replace 
% launchdpad \\
% https://review.openstack.org/ & Patch and review system capable of
% verifying patches using unit \& functional tests \\
% https://review.opendev.org/ & Replacement for review.openstack.org introduced
% around May 2019 \\
% https://github.com/openstack/watcher & Repository hosting the Watcher source
% code \\
% https://opendev.org/openstack/watcher & New repository for hosting source code
% introduced in May 2019 \\
% https://governance.openstack.org/ & Information how a set of organizational
% bodies organizes the governance \\
% \bottomrule
% \end{tabular}
% \begin{tablenotes}[para,flushleft]
%      \centering This overview lists all online resources used to analyze how
%      watcher is governed and collaborates.
% \end{tablenotes}
% \end{threeparttable}
% \end{adjustbox}
% \end{table}

% \subsection*{Meeting suggestion email}
% \addcontentsline{toc}{subsection}{\protect\numberline{}Meeting suggestion email}
% \label{appendix:suggestionemail}

% \noindent Hello everyone,

% \noindent We would like to propose reintroducing the meetings after the release
% of \\ Stein. I feel meetings are important as it allows us to discuss what we \\
% want to work on and what problems we are experiencing with regards to \\
% Watcher. A notable example of a topic that I feel we should address is \\
% the deadlock in python 3.7 with unit tests. \\

% \noindent Currently we have a meeting scheduled every week but in different \\ 
% channels at different times, however, the documentation is unclear about \\
% the exact time and channel as the following two web-pages disagree: \\
% https://docs.openstack.org/watcher/latest/contributor/contributing.html \\
% https://wiki.openstack.org/wiki/Watcher \\

% \noindent  I would like to propose a meeting on Wednesday at 08:00 UTC on
% even \\ weeks in \#openstack-meeting-4 as I feel this time work bests for
% our \\ mixed time-zones. \\

% \noindent I look forward to hearing from you. \\

% \noindent Kind Regards, \\
% \noindent Corne Lukken \\

% \subsection*{Watcher Drivers team email}
% \addcontentsline{toc}{subsection}{\protect\numberline{}Watcher Drivers team email}
% \label{appendix:driveremail}

% \noindent Done. \\

% \noindent Have a nice day!  Thanks for contributing on Watcher ! \\

% \noindent Jean-Emile
% \noindent DARTOIS \\

% \noindent \{P\}     PhD Student
% \noindent Cloud Computing \\

% \noindent \{T\}     +33 (0) 2 56 35 8260 \\
% \noindent \{W\}     www.b-com.com

% \noindent \_\_\_\_\_\_\_\_\_\_\_\_\_\_\_\_\_\_\_\_\_\_\_\_\_\_\_\_\_\_\_\_\_\_\_\_\_\_\_\_\\
% \noindent De : info@dantalion.nl <info@dantalion.nl> \\
% \noindent Envoyé : jeudi 13 juin 2019 10:23 \\
% \noindent À : Jean-Émile DARTOIS \\
% \noindent Objet : Re: The Watcher Drivers team seems poorly maintained \\

% \noindent Hello, \\

% \noindent Current PTL is Licanwei if you can make him administrator than we \\
% have regained control that would be much appreciated. \\

% \noindent Kind regards, \\
% \noindent Corne Lukken (Dantali0n) \\

% \noindent On 6/13/19 10:00 AM, Jean-Émile DARTOIS wrote: \\
% \noindent > Hello, \\
% \noindent > \\
% \noindent > Sorry, I'm not involved anymore in Watcher. I added you as an \\
% approved member. Who is the current PTL? I can make him administrator. \\
% \noindent > \\
% \noindent > Best regards, \\
% \noindent > \\
% \noindent > \\
% \noindent > Jean-Emile \\
% \noindent > DARTOIS \\
% \noindent > \\
% \noindent > \{P\}     PhD Student \\
% \noindent > Cloud Computing \\
% \noindent > \\
% \noindent > \{T\}     +33 (0) 2 56 35 8260 \\
% \noindent > \{W\}     www.b-com.com \\
% \noindent > \\
% \noindent > \_\_\_\_\_\_\_\_\_\_\_\_\_\_\_\_\_\_\_\_\_\_\_\_\_\_\_\_\_\_\_\_\_\_\_\_\_\_\_\_\\
% \noindent > De : bounces@canonical.com <bounces@canonical.com> de la part de \\
% Dantali0n < info@dantalion.nl> \\
% \noindent > Envoyé : jeudi 13 juin 2019 09:18 \\
% \noindent > À : Jean-Émile DARTOIS \\
% \noindent > Objet : The Watcher Drivers team seems poorly maintained \\
% \noindent > \\
% \noindent > Hello, \\
% \noindent > \\
% \noindent > It seems like the administrators in the Watcher Drivers team have \\
% not be \\
% \noindent > contributing for a long time. I and several others have requested \\
% \noindent > membership months ago without a response. \\
% \noindent > \\
% \noindent > I was hoping you could make  me or Licanwei administrator so we can \\
% \noindent > regain control over the managed of this team. \\
% \noindent > \\
% \noindent > Kind regards, \\
% \noindent > Dantali0n \\
% \noindent > --\\
% \noindent > This message was sent from Launchpad by \\
% \noindent > Dantali0n (https://launchpad.net/~dantalion) \\
% \noindent > using the "Contact this team's admins" link on the Watcher Drivers\\
% team page \\
% \noindent > (https://launchpad.net/~watcher-drivers). \\
% \noindent > For more information see \\
% \noindent > https://help.launchpad.net/YourAccount/ContactingPeople \\

\end{document}